% Copyright 2013 Christophe-Marie Duquesne <chmd@chmd.fr>
% Copyright 2014 Mark Szepieniec <http://github.com/mszep>
% 
% ConText style for making a resume with pandoc. Inspired by moderncv.
% 
% This CSS document is delivered to you under the CC BY-SA 3.0 License.
% https://creativecommons.org/licenses/by-sa/3.0/deed.en_US

\startmode[*mkii]
  \enableregime[utf-8]  
  \setupcolors[state=start]
\stopmode

\setupcolor[hex]
\definecolor[titlegrey][h=757575]
\definecolor[sectioncolor][h=397249]
\definecolor[rulecolor][h=9cb770]

% Enable hyperlinks
\setupinteraction[state=start, color=sectioncolor]

\setuppapersize [A4][A4]
\setuplayout    [width=middle, height=middle,
                 backspace=20mm, cutspace=0mm,
                 topspace=10mm, bottomspace=20mm,
                 header=0mm, footer=0mm]

%\setuppagenumbering[location={footer,center}]

\setupbodyfont[11pt, helvetica]

\setupwhitespace[medium]

\setupblackrules[width=31mm, color=rulecolor]

\setuphead[chapter]      [style=\tfd]
\setuphead[section]      [style=\tfd\bf, color=titlegrey, align=middle]
\setuphead[subsection]   [style=\tfb\bf, color=sectioncolor, align=right,
                          before={\leavevmode\blackrule\hspace}]
\setuphead[subsubsection][style=\bf]

\setuphead[chapter, section, subsection, subsubsection][number=no]

%\setupdescriptions[width=10mm]

\definedescription
  [description]
  [headstyle=bold, style=normal,
   location=hanging, width=18mm, distance=14mm, margin=0cm]

\setupitemize[autointro, packed]    % prevent orphan list intro
\setupitemize[indentnext=no]

\defineitemgroup[enumerate]
\setupenumerate[each][fit][itemalign=left,distance=.5em,style={\feature[+][default:tnum]}]

\setupfloat[figure][default={here,nonumber}]
\setupfloat[table][default={here,nonumber}]

\setuptables[textwidth=max, HL=none]
\setupxtable[frame=off,option={stretch,width}]

\setupthinrules[width=15em] % width of horizontal rules

\setupdelimitedtext
  [blockquote]
  [before={\setupalign[middle]},
   indentnext=no,
  ]


\starttext

\section[title={Nicolas Bayona},reference={nicolas-bayona}]

\subsection[title={> Systems engineering student passionate about
technological stuff and the use of technology to improve people's
lives.},reference={systems-engineering-student-passionate-about-technological-stuff-and-the-use-of-technology-to-improve-peoples-lives.}]

\subsection[title={Education},reference={education}]

2019-2023 (expected):\crlf
{\bf BSc, Systems engineering}; Pontifical Xaverian University (Bogota,
Colombia)

\startitemize[packed]
\item
  Final project title: Still not sure :C
\stopitemize

\subsection[title={Experience},reference={experience}]

{\bf Work Experience:}

{\em {\bf Teacher assistant of Introduction to programming at Pontifical
Xaverian University}}:

\startitemize
\item
  As a teacher assistant at the university I was in charge of:

  \startitemize[packed]
  \item
    Explain relevant programming concepts to students.
  \item
    Help students understand the logical process behind programming.
  \item
    Tell students about important lessons I learnt throughout my life at
    programming.
  \stopitemize
\stopitemize

{\bf Volunteer Work Experience:}

{\em {\bf Academic mentor at Pontifical Xaverian University}}:

\startitemize
\item
  As an academic mentor at the university I was in charge of:

  \startitemize[packed]
  \item
    Welcoming new coming students to a new stage in their lives.
  \item
    Helping the new coming students in their adaptation process.
  \item
    Tell students about my experience as an student so far, my greatest
    challenges and important lessons I learnt throughout my life.
  \item
    Act as a leader to new coming students, so that they felt they were
    accompanied in their adaptation processes.
  \stopitemize
\stopitemize

\subsection[title={Technical
Experience},reference={technical-experience}]

Most of my technical work is in my GitHub profile, you can visit my
profile at
\useURL[url1][https://github.com/nclsbayona][][github.com/nclsbayona]\from[url1]

{\bf My Cool Side Projects:}\crlf
Usually I practice my skills via some helpful project that I tend to
mantain over time, I think it's really good to automate repetitive tasks
and jobs a bot or some type of automation software can do on its own.
Examples of this are most of my GitHub repositories:

\startitemize[packed]
\item
  {\bf {\em \useURL[url2][https:github.com/nclsbayona/nclsbayona][][My
  Profile README]\from[url2]:}} My GitHub profile's readme is an example
  of a programming and automation project I've realized.
\item
  {\bf {\em \useURL[url3][https:github.com/nclsbayona/nclsbayona.github.io][][My
  WebPage]\from[url3]:}} I recently developed my own website which acts
  as a showcase for my abilities as web developer.
\stopitemize

{\bf Programming Languages:}\crlf
Throughout my life, I've worked in many projects and therefore, I've
adquired a lot of experience in various programming languages. Next,
I'll name a few and some relevant stuff I learnt from them.

\startitemize
\item
  {\bf {\em C++:}}\crlf
  The first programming language I learnt was C++, the most valuable
  thing I think I learnt from my C++ learning was programming logic. I
  think that C++ is a really great language because it acts as a basis
  for other languages and many tools that have been working flawlessly
  since it first appeared almost 40 years ago.
\item
  {\bf {\em Python:}}\crlf
  After learning C++, I decided to learn Python, because its reputation
  was great, it simplified the development process as its syntax was
  similar to natural language and included really powerful mathematical
  tools that helped to perform tasks that might involve mathematical
  stuff.
\item
  {\bf {\em Java:}}\crlf
  After learning Python, I learnt Java since it was a more mature
  language, Java was highly used in a lot of enterprise-scale
  applications and many cool frameworks like Spring Framework where
  developed considering Java as a first-class citizen which gave it an
  advantage.
\item
  {\bf {\em Other relevant information:}}\crlf
  Also, I have basic knowledge of other programming languages like
  {\em {\bf TypeScript}, {\bf JavaScript}, {\bf C}, {\bf Dart}} and
  {\em {\bf Swift}}, as well as markup languages such as
  {\em {\bf YAML}, {\bf HTML}, , {\bf Markdown}, {\bf XML}} and
  {\em {\bf JSON}}, and other tech-related tools like {\em {\bf Git}},
  {\bf Kubernetes - Kubectl}, {\bf Docker - Podman}, {\bf Github
  Actions}_ and {\em {\bf AWS}}.
\stopitemize

\subsection[title={Aditional
information},reference={aditional-information}]

\startitemize[packed]
\item
  Languages:
  \startitemize[packed]
  \item
    Spanish (Native speaker)
  \item
    English (Fully-conversational)
  \stopitemize
\stopitemize

\subsection[title={Awards},reference={awards}]

\startitemize[packed]
\item
  Academic excellence
  \startitemize[packed]
  \item
    August 2020
  \stopitemize
\item
  Academic excellence
  \startitemize[packed]
  \item
    January 2021
  \stopitemize
\item
  Academic excellence
  \startitemize[packed]
  \item
    August 2021
  \stopitemize
\item
  Academic excellence
  \startitemize[packed]
  \item
    January 2022
  \stopitemize
\stopitemize

\thinrule

\useURL[url4][mailto:bayona.n@javeriana.edu.co][][bayona.n@javeriana.edu.co]\from[url4]
• Bogota, Colombia\crlf
Reach me on \useURL[url5][https://t.me/nclsbayona][][Telegram, my
username is nclsbayona]\from[url5].

\thinrule

\stoptext
