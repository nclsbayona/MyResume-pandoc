% Copyright 2013 Christophe-Marie Duquesne <chmd@chmd.fr>
% Copyright 2014 Mark Szepieniec <http://github.com/mszep>
% 
% ConText style for making a resume with pandoc. Inspired by moderncv.
% 
% This CSS document is delivered to you under the CC BY-SA 3.0 License.
% https://creativecommons.org/licenses/by-sa/3.0/deed.en_US

\startmode[*mkii]
  \enableregime[utf-8]  
  \setupcolors[state=start]
\stopmode

\setupcolor[hex]
\definecolor[titlegrey][h=757575]
\definecolor[sectioncolor][h=397249]
\definecolor[rulecolor][h=9cb770]

% Enable hyperlinks
\setupinteraction[state=start, color=sectioncolor]

\setuppapersize [A4][A4]
\setuplayout    [width=middle, height=middle,
                 backspace=20mm, cutspace=0mm,
                 topspace=10mm, bottomspace=20mm,
                 header=0mm, footer=0mm]

%\setuppagenumbering[location={footer,center}]

\setupbodyfont[11pt, helvetica]

\setupwhitespace[medium]

\setupblackrules[width=31mm, color=rulecolor]

\setuphead[chapter]      [style=\tfd]
\setuphead[section]      [style=\tfd\bf, color=titlegrey, align=middle]
\setuphead[subsection]   [style=\tfb\bf, color=sectioncolor, align=right,
                          before={\leavevmode\blackrule\hspace}]
\setuphead[subsubsection][style=\bf]

\setuphead[chapter, section, subsection, subsubsection][number=no]

%\setupdescriptions[width=10mm]

\definedescription
  [description]
  [headstyle=bold, style=normal,
   location=hanging, width=18mm, distance=14mm, margin=0cm]

\setupitemize[autointro, packed]    % prevent orphan list intro
\setupitemize[indentnext=no]

\defineitemgroup[enumerate]
\setupenumerate[each][fit][itemalign=left,distance=.5em,style={\feature[+][default:tnum]}]

\setupfloat[figure][default={here,nonumber}]
\setupfloat[table][default={here,nonumber}]

\setuptables[textwidth=max, HL=none]
\setupxtable[frame=off,option={stretch,width}]

\setupthinrules[width=15em] % width of horizontal rules

\setupdelimitedtext
  [blockquote]
  [before={\setupalign[middle]},
   indentnext=no,
  ]


\starttext

\section[title={Nicolas Bayona},reference={nicolas-bayona}]

\subsection[title={> Systems engineering student passionate about
technological stuff and the use of technology to improve people's
lives.},reference={systems-engineering-student-passionate-about-technological-stuff-and-the-use-of-technology-to-improve-peoples-lives.}]

\subsection[title={Education},reference={education}]

2019-2023 (expected):\crlf
{\bf BSc, Systems engineering}; Pontifical Xaverian University (Bogota,
Colombia)

\subsection[title={Experience},reference={experience}]

{\bf Work Experience:}

\startitemize[packed]
\item
  {\em Teacher assistant of Introduction to programming at Pontifical
  Xaverian University}
\stopitemize

{\bf Volunteer Work Experience:}

\startitemize[packed]
\item
  {\em Academic mentor at Pontifical Xaverian University}
\stopitemize

\subsection[title={Technical
Experience},reference={technical-experience}]

Most of my technical work is in my GitHub profile, you can visit my
profile at \useURL[url1][https://github.com/nclsbayona]\from[url1]

\subsection[title={{\bf My Cool Side
Projects:}},reference={my-cool-side-projects}]

Usually I practice my skills via some helpful project that I tend to
mantain over time, I think it's really good to automate repetitive tasks
and jobs a bot or some type of automation software can do on its own.
Examples of this are most of my GitHub repositories:

\startitemize[packed]
\item
  {\bf {\em \useURL[url2][https:github.com/nclsbayona/nclsbayona][][My
  Profile README]\from[url2]:}} My GitHub profile's readme is an example
  of a programming and automation project I've realized.
\item
  {\bf {\em \useURL[url3][https:github.com/nclsbayona/nclsbayona.github.io][][My
  WebPage]\from[url3]:}} I recently developed my own website which acts
  as a showcase for my habilities as web developer.
\stopitemize

{\bf Programming Languages:}\crlf
Throughout my life, I've worked in many projects and therefore, I've
adquired a lot of experience in various programming languages. Next,
I'll name a few and some relevant stuff I learnt from them.

\startitemize[packed]
\item
  {\bf {\em C++}}\crlf
\item
  {\bf {\em Python}}\crlf
\item
  {\bf {\em Java}}\crlf
\item
  {\bf {\em Other relevant tools:}}\crlf
  \startitemize[packed]
  \item
    Git, Kubernetes - Kubectl, Docker - Podman, Github Actions {\em and}
    AWS.
  \stopitemize
\stopitemize

\subsection[title={Aditional
information},reference={aditional-information}]

\startitemize[packed]
\item
  Languages:
  \startitemize[packed]
  \item
    Spanish (Native speaker)
  \item
    English (Fully-conversational)
  \stopitemize
\stopitemize

\subsection[title={Awards},reference={awards}]

\startitemize[packed]
\item
  Academic excellence
  \startitemize[packed]
  \item
    August 2020
  \stopitemize
\item
  Academic excellence
  \startitemize[packed]
  \item
    January 2021
  \stopitemize
\item
  Academic excellence
  \startitemize[packed]
  \item
    August 2021
  \stopitemize
\item
  Academic excellence
  \startitemize[packed]
  \item
    January 2022
  \stopitemize
\item
  Academic excellence
  \startitemize[packed]
  \item
    August 2022
  \stopitemize
\stopitemize

\thinrule

\useURL[url4][mailto:bayona.n@javeriana.edu.co][][bayona.n@javeriana.edu.co]\from[url4]
• Bogota, Colombia\crlf
Reach me on \useURL[url5][https://t.me/nclsbayona][][Telegram, my
username is nclsbayona]\from[url5].

\thinrule

\stoptext
